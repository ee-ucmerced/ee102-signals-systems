\makeatletter
\def\input@path{{../styles/}{../../styles/}{../../../styles/}{../}{../../}{../../../}}
\makeatother

\documentclass{ee102_pset}
% macros.tex - Course meta information
\renewcommand{\course}{EE 102}
\renewcommand{\coursetitle}{Signal Processing and Linear Systems}
\renewcommand{\instructor}{Ayush Pandey}
\renewcommand{\semester}{Fall}
\renewcommand{\year}{2025}
\renewcommand{\shorttitle}{Week 1: Introduction to Signals}
% Use \renewcommand to avoid 'already defined' errors

% The following packages can be found on http:\\www.ctan.org
% \usepackage{graphics} % for pdf, bitmapped graphics files
%\usepackage{epsfig} % for postscript graphics files
%\usepackage{mathptmx} % assumes new font selection scheme installed
%\usepackage{times} % assumes new font selection scheme installed
\usepackage{amsmath} % assumes amsmath package installed
\usepackage{amssymb,mathtools}  % assumes amsmath package installed
\usepackage{xcolor}
\usepackage{pgfplots,subcaption}
\usepackage[hidelinks]{hyperref}
\usepackage{verbatim}
\usepackage{graphicx}
\usepackage{listings}

% -------- listings (Python) ----------
\lstdefinestyle{py}{
  language=Python,
  basicstyle=\ttfamily\small,
  keywordstyle=\color{blue!60!black}\bfseries,
  commentstyle=\color{green!40!black},
  stringstyle=\color{orange!60!black},
  showstringspaces=false,
  columns=fullflexible,
  frame=single,
  framerule=0.3pt,
  numbers=left,
  numberstyle=\tiny,
  xleftmargin=1em,
  tabsize=2,
  breaklines=true,
}
\usepackage[american]{circuitikz}
\usepackage{tikz}
\usepackage{caption}    
\usepackage{lscape}
\usepackage{soul}
\usepackage{tikz}
\usetikzlibrary{calc,angles,quotes,arrows.meta}

\usepackage{hyperref}
\hypersetup{
    colorlinks=true,
    linkcolor=blue,
    filecolor=magenta,      
    urlcolor=blue,
    pdftitle={week1_notes},
    pdfpagemode=FullScreen,
}
%\usepackage{float} 

%\usepackage[demo]{graphicx}
\pgfplotsset{compat=1.18}
% \usepgfplotslibrary{fillbetween}

\newsavebox{\measurebox}

\let\proof\relax\let\endproof\relax


\newcommand{\norm}[1]{\left\lVert#1\right\rVert}
\def\abs#1{\left\lvert#1\right\rvert}
\let\proof\relax
\let\endproof\relax
\usepackage{amsthm}
\usepackage{accents}
\usepackage{relsize}
\newcommand{\ubar}[1]{\underaccent{\bar}{#1}}
\newtheorem{theorem}{Theorem}
\newtheorem{corollary}{Corollary}[theorem]
\newtheorem{lemma}{Lemma}
\newtheorem{proposition}{Proposition}
\newtheorem{statement}{Statement}

\theoremstyle{definition}
\newtheorem{definition}{Definition}
 
\theoremstyle{remark}
\newtheorem*{remark}{Remark}
\theoremstyle{remark}
\newtheorem*{claim}{Claim}
\setlength{\parindent}{0cm}
\newenvironment{nalign}{
    \begin{equation}
    \begin{aligned}
}{
    \end{aligned}
    \end{equation}
    \ignorespacesafterend
}


% Assignment info
\author{\rule{3cm}{0.4pt}} % Name placeholder
\submitdate{\rule{3cm}{0.4pt}} % Submission date placeholder
\problemset{Homework \#9: Applications of Fourier Transform}
% renew instructor to be empty
\renewcommand{\instructor}{}
\renewcommand{\duedate}{November 3, 2025}
\shorttitle{Homework \#9}

\begin{document}

\problem{1}

Record yourself! For this homework, create a short audio recording (1-2 minutes) where you explain one of the key concepts you learned in this course so far. You can choose any topic that interests you, such as Fourier series, Fourier transforms, convolution, or any other topic covered in class. 

Important: Make sure to vary your pitch, tone, and excitement level to get a rich audio signal. Save/convert this audio file as a .wav file so that you can read and manipulate it in Python/MATLAB.

\problempart [10 points]
Load your audio file in Python/MATLAB and plot the time-domain signal. Describe any interesting features you observe in the waveform (just by visually looking at it).

\problempart [40 points]
Compute the discrete Fourier transform (DFT) of your audio signal. Plot the magnitude spectrum of the DFT and identify the dominant frequency components present in your recording. 

Important: For this problem set, you are not allowed to use built-in FFT functions. You must implement the DFT computation manually using the DFT formula. You may approximate the DFT by computing it only at a subset of frequencies.

\problempart [30 points]
Design your personal audio-based password: Choose a specific sequence of frequencies (at least 3 distinct frequencies) that will serve as your audio password. Using your DFT results, modify your original audio signal to enhance these chosen frequencies while suppressing others. You can achieve this by applying a frequency-domain filter (recall the kinds of filter designs we discussed in class).

Next, reconstruct the modified audio signal back to the time domain using the inverse DFT (again, implement this manually without built-in functions). 

Finally, save and listen to your modified audio signal. 

\problempart [20 points] Ask a friend for their recorded audio file (.wav) and show that your password works by applying the same frequency-domain filter to their audio signal and showing that your friend's audio does not match up with your password frequencies.

Visualize the frequency domain representation (filtered) for both your audio and your friend's audio to demonstrate the difference.

\end{document}
