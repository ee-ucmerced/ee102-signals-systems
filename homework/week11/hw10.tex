\makeatletter
\def\input@path{{../styles/}{../../styles/}{../../../styles/}{../}{../../}{../../../}}
\makeatother

\documentclass{ee102_pset}
% macros.tex - Course meta information
\renewcommand{\course}{EE 102}
\renewcommand{\coursetitle}{Signal Processing and Linear Systems}
\renewcommand{\instructor}{Ayush Pandey}
\renewcommand{\semester}{Fall}
\renewcommand{\year}{2025}
\renewcommand{\shorttitle}{Week 1: Introduction to Signals}
% Use \renewcommand to avoid 'already defined' errors

% The following packages can be found on http:\\www.ctan.org
% \usepackage{graphics} % for pdf, bitmapped graphics files
%\usepackage{epsfig} % for postscript graphics files
%\usepackage{mathptmx} % assumes new font selection scheme installed
%\usepackage{times} % assumes new font selection scheme installed
\usepackage{amsmath} % assumes amsmath package installed
\usepackage{amssymb,mathtools}  % assumes amsmath package installed
\usepackage{xcolor}
\usepackage{pgfplots,subcaption}
\usepackage[hidelinks]{hyperref}
\usepackage{verbatim}
\usepackage{graphicx}
\usepackage{listings}

% -------- listings (Python) ----------
\lstdefinestyle{py}{
  language=Python,
  basicstyle=\ttfamily\small,
  keywordstyle=\color{blue!60!black}\bfseries,
  commentstyle=\color{green!40!black},
  stringstyle=\color{orange!60!black},
  showstringspaces=false,
  columns=fullflexible,
  frame=single,
  framerule=0.3pt,
  numbers=left,
  numberstyle=\tiny,
  xleftmargin=1em,
  tabsize=2,
  breaklines=true,
}
\usepackage[american]{circuitikz}
\usepackage{tikz}
\usepackage{caption}    
\usepackage{lscape}
\usepackage{soul}
\usepackage{tikz}
\usetikzlibrary{calc,angles,quotes,arrows.meta}

\usepackage{hyperref}
\hypersetup{
    colorlinks=true,
    linkcolor=blue,
    filecolor=magenta,      
    urlcolor=blue,
    pdftitle={week1_notes},
    pdfpagemode=FullScreen,
}
%\usepackage{float} 

%\usepackage[demo]{graphicx}
\pgfplotsset{compat=1.18}
% \usepgfplotslibrary{fillbetween}

\newsavebox{\measurebox}

\let\proof\relax\let\endproof\relax


\newcommand{\norm}[1]{\left\lVert#1\right\rVert}
\def\abs#1{\left\lvert#1\right\rvert}
\let\proof\relax
\let\endproof\relax
\usepackage{amsthm}
\usepackage{accents}
\usepackage{relsize}
\newcommand{\ubar}[1]{\underaccent{\bar}{#1}}
\newtheorem{theorem}{Theorem}
\newtheorem{corollary}{Corollary}[theorem]
\newtheorem{lemma}{Lemma}
\newtheorem{proposition}{Proposition}
\newtheorem{statement}{Statement}

\theoremstyle{definition}
\newtheorem{definition}{Definition}
 
\theoremstyle{remark}
\newtheorem*{remark}{Remark}
\theoremstyle{remark}
\newtheorem*{claim}{Claim}
\setlength{\parindent}{0cm}
\newenvironment{nalign}{
    \begin{equation}
    \begin{aligned}
}{
    \end{aligned}
    \end{equation}
    \ignorespacesafterend
}


% Assignment info
\author{\rule{3cm}{0.4pt}} % Name placeholder
\submitdate{\rule{3cm}{0.4pt}} % Submission date placeholder
\problemset{Homework \#10: Filter Design}
% renew instructor to be empty
\renewcommand{\instructor}{}
\renewcommand{\duedate}{November 17, 2025}
\shorttitle{Homework \#10}

\begin{document}

\problem{1} An underdamped series RLC circuit is excited by an input at time \(t_0\). The circuit outputs a voltage that is a damped oscillation at its natural frequency. This output signal is modeled as
\[
x(t)=A\,e^{-\alpha\,(t-t_0)}\cos\!\big(\omega_0\,(t-t_0)\big)\,u(t-t_0),
\qquad
\alpha>0,\ \omega_0>0,
\]
where \(u(\cdot)\) is the unit step. 

\problempart \textbf{(3 points)}
Write two time-domain properties of \(x(t)\) that you can observe directly from the signal equation or by graphing it. You may graph the signal using any software (Python, MATLAB, Desmos) or by hand.

\problempart \textbf{(2 points)} Write a signal $\bar{x}(t)$ that when shifted by \(t_0\) gives \(x(t)\).

\problempart \textbf{(10 points)}
Compute the Fourier transform \(X(\omega)\) of \(x(t)\).

Hint: Use Euler's formula to express the cosine as a sum of complex exponentials --- don't solve integration by parts!

\problempart \textbf{(5 points)} Compute the Fourier transform $\bar X(\omega)$ of $\bar x(t)$. Using the time-shift property, prove that \(X(\omega) = e^{-j\omega t_0} \bar{X}(\omega)\) to verify your result.


% \[ X(\omega)=\frac{A}{2}\,e^{-j\omega t_0}\!\left(\frac{1}{\alpha+j(\omega-\omega_0)}+\frac{1}{\alpha+j(\omega+\omega_0)}\right).
% \]

% \problempart \textbf{(5 points)} Sketch \(|X(\omega)|\). Indicate the spectral peaks (the frequency peaks) at \(\pm\omega_0\).

\problempart \textbf{(5 points)}
Define an ideal low-pass filter as follows:
\[
H_{\mathrm{LP}}(\omega)=
\begin{cases}
1,& |\omega|\le \Omega_c,\\
0,& |\omega|>\Omega_c.
\end{cases}
\]
Sketch \(H_{\mathrm{LP}}(\omega)\) for positive and negative \(\omega\). Label \(\pm\Omega_c\) on your graph.

\problempart \textbf{(5 points)}
By using the inverse Fourier transform, find the impulse response of the low-pass filter. 

% \[
% h_{\mathrm{LP}}(t)=\frac{1}{2\pi}\!\int_{-\Omega_c}^{\Omega_c} e^{j\omega t}\,d\omega
% =\frac{\sin(\Omega_c t)}{\pi t}.
% \]

\textbf{Hint:} use even symmetry of \(H_{\mathrm{LP}}(\omega)\). Solved example 4.5 in Oppenheim \& Willsky 2nd Edition has the answer. It is OK to just use the result and approach from the book.

\problempart \textbf{(10 points)}
We know that if we `apply' the filter system to the input. That is, if there is a system with impulse response \(h_{\mathrm{LP}}(t)\), and we apply an input $x(t)$ to this system to obtain an output $y(t)$. Using the convolution theorem, we know that in the frequency domain, we have that \(Y(\omega)=X(\omega)\,H_{\mathrm{LP}}(\omega)\). Compute $Y(\omega)$ and discuss the properties of $Y(\omega)$. 

\problempart \textbf{(10 points)}
Now, let's do the same in time-domain. Write the convolution integral
\[
y(t)=(x*h_{\mathrm{LP}})(t)=\int_{-\infty}^{\infty} x(\tau)\,h_{\mathrm{LP}}(t-\tau)\,d\tau.
\]
Solve the convolution integral to compute $y(t)$ in time-domain.
Compute the inverse Fourier transform of \(Y(\omega)\) to obtain \(y(t)\) and verify your result. 

\textbf{\color{blue}Important:} You will not be able to solve the convolution integral in closed-form. Instead, express your answer as an integral expression that can be evaluated numerically. Things can be quite difficult in time-domain!

\problempart \textbf{(15 points)}
Using Parseval's theorem, we can define a new type of energy in the frequency domain that is band-limited to \(|\omega|\le \Omega_c\):
\[
E_{\mathrm{LP}}=\frac{1}{2\pi}\int_{-\Omega_c}^{\Omega_c} |X(\omega)|^2\,d\omega,
\]
this is the energy contained in the signal \(x(t)\) that lies within the passband of the low-pass filter.

Compute \(E_{\mathrm{LP}}\) in terms of \(\alpha,\omega_0,\Omega_c\). 

\textbf{Hint:} use \(\big|\alpha+j(\omega-\omega_0)\big|^{-2}=\frac{1}{\alpha^2+(\omega-\omega_0)^2}\) and \(\int \frac{d\omega}{a^2+(\omega-b)^2}=\frac{1}{a}\arctan\!\frac{\omega-b}{a}\). You may find it useful to ignore cross terms if \(\alpha\ll \omega_0\). How do you justify this approximation physically?

\problempart \textbf{(10 points)}
Explore other possible filters in frequency domain: low-pass, band-pass, and high-pass. Comment on which one might better isolate the ring-down mode in the RLC circuit. Refer to sketches, \(h(t)\), and band-energy calculations. You don't need to solve it out completely but make convincing arguments using Fourier analysis properties.


\problem{2} Validate Problem 1 using computer programs. It is \textbf{highly recommended that you start early with the homework so that you can get help during TA's Friday office hour}.

With a sampling rate \(F_s\) Hz, we have a time interval of \(\Delta t=1/F_s\) and we can write the continuous-time signal above by substituting \(t=n\Delta t\) as a discrete-time signal:
\[
x[n]=A\,e^{-\alpha(n-n_0)\Delta t}\cos\!\big(\omega_0 (n-n_0)\Delta t\big)\,u[n-n_0].
\]

\problempart \textbf{(5 points)}
Choose parameters \((A,\alpha,\omega_0,t_0)\) and \(F_s\). Implement the equation above in a function. Then, plot \(x[n]\) versus \(n\Delta t\). Label your graph.
Make sure to choose $F_s$ to be greater than \(2 \cdot \frac{\omega_0}{2\pi}\). You will learn why this is important next week. 

For a simple start, set $F_s = 2000$ Hz and ensure that the condition above is satisfied.

\problempart \textbf{(10 points)}
Re-use the previous homework assignment code for DFT implementation to compute the DFT with length \(N\) for the signal above (it is also OK to use FFT libraries). 

Helpful starting point:

Set $N = 2048$ and $L = 8192$. If your generated \(x[n]\) has length \(<N\), zero-pad it to length \(N\). If it is longer, keep the first \(N\) samples. Then zero-pad from \(N\) to \(L\) before taking the DFT. Remember that with DFT, the frequency is sampled at discrete frequencies. Use the following frequency grid:
\[
\omega_k=\frac{2\pi k F_s}{L},\quad k=0,1,\dots,L-1.
\]
The index corresponding to \(\omega_0\) will be 
\[
k_0=\operatorname{round}\!\left(\frac{\omega_0\,L}{2\pi F_s}\right).
\]

Plot \(|X[k]|\) versus discrete frequency. Indicate indices corresponding to \(\pm\omega_0\). You may need to center the spectrum for better visualization (using \texttt{fftshift}).

\problempart \textbf{(10 points)}
Implement the ideal low-pass filter $H[k]$. Compute \(y[n]\) using inverse DFT. Plot the filtered signal \(y[n]\).

The discrete-time DFT mask is
\[
H[k]=\begin{cases}
1,& |\omega_k|\le \Omega_c,\\
0,& \text{otherwise}.
\end{cases}
\]
You may start by setting \(\Omega_c = 0.8\,\omega_0\).

Compute \(Y[k]=X[k]\cdot H[k]\). Compute \(y[n]\) by the \(L\)-point inverse DFT. Plot \(\operatorname{Re}\{y[n]\}\) versus \(n\Delta t\).

\textbf{\color{blue}Optional exploration (not graded):}

\problempart Implement at least one other filter type (high-pass or band-pass, or something else of your choice). Provide the DFT mask, impulse response, and plot the filtered signal. Comment on how well the filter isolates the ring-down mode in the RLC circuit.


\problempart Estimate band-limited input energy by summing \(|X[k]|^2\) over passband indices to verify your answer in Problem 1. Find out how well your filter isolates the ringing mode for your chosen parameters. Experiment with different parameters.

\end{document}


% Replace your Problem 2 block with this fully specified version.

\problem{2} Validate Problem 1 using computer programs. You may start on the coding problem before finishing Problem 1.

With a sampling rate \(F_s\) Hz, we have \(\Delta t=1/F_s\) and the discrete-time signal
\[
x[n]=A\,e^{-\alpha(n-n_0)\Delta t}\cos\!\big(\omega_0 (n-n_0)\Delta t\big)\,u[n-n_0].
\]

\problempart \textbf{(5 points)}
Choose \(A,\alpha,\omega_0,t_0\) and \(F_s\). Plot \(x[n]\) versus \(n\Delta t\). 
\textbf{Use exactly:} \(F_s=2000~\text{Hz}\). Ensure \(F_s>2\cdot \frac{\omega_0}{2\pi}\) (Nyquist).

\problempart \textbf{(10 points)}
Compute the DFT of \(x[n]\) with \textbf{fixed} sizes and padding.

\textbf{Use these exact sizes:}
\[
N=2048,\qquad L=8192.
\]
If your generated \(x[n]\) has length \(<N\), zero-pad it to length \(N\). If it is longer, keep the first \(N\) samples. Then zero-pad from \(N\) to \(L\) before taking the DFT.

\textbf{Frequency grid (use exactly this):}
\[
\omega_k=\frac{2\pi k F_s}{L},\quad k=0,1,\dots,L-1.
\]
Index corresponding to \(\omega_0\):
\[
k_0=\operatorname{round}\!\left(\frac{\omega_0\,L}{2\pi F_s}\right).
\]

Plot \(|X[k]|\) versus \(\omega_k\). For visualization only, you may apply \texttt{fftshift} to both \(X\) and \(\omega_k\) before plotting. Clearly mark the bins nearest \(\pm\omega_0\).

\problempart \textbf{(10 points)}
Implement an \textbf{ideal low-pass} in the DFT domain with a \textbf{fixed} cutoff tied to your \(\omega_0\).

\textbf{Use exactly:} \(\Omega_c = 0.8\,\omega_0\).

Form the mask
\[
H[k]=\begin{cases}
1,& |\omega_k|\le \Omega_c,\\
0,& \text{otherwise}.
\end{cases}
\]
Compute \(Y[k]=X[k]\cdot H[k]\). Compute \(y[n]\) by the \(L\)-point inverse DFT. Plot \(\operatorname{Re}\{y[n]\}\) versus \(n\Delta t\). In one sentence, state what the filter kept and what it removed.

\textbf{\color{blue}Optional (not graded):}
\problempart Implement a band-pass with \(\Omega_1=0.9\,\omega_0\), \(\Omega_2=1.1\,\omega_0\) using the same \(L\) and grid. Plot the result.

\problempart Numerically estimate band-limited energy (no algebra):
\[
E_{\mathrm{LP}} \approx \frac{1}{2\pi}\sum_{\{k:\,|\omega_k|\le \Omega_c\}} |X[k]|^2\,\Delta\omega,
\qquad \Delta\omega=\frac{2\pi F_s}{L}.
\]
Report the value. Compare to the band-pass value if you did the optional part.

% --- One-line, working convolution-theorem check (keep it short) ---
\textbf{DFT check (time filtering = frequency multiplication).}
Use the same \(L=8192\). Let \(h[n]\) be any FIR of length \(\mathbf{M=129}\) (e.g., your ideal LP impulse windowed to 129 taps). Zero-pad \(x[n]\) to length \(L\) and \(h[n]\) to length \(L\). Compute
\[
X=\mathrm{DFT}_L(x),\quad H=\mathrm{DFT}_L(h),\quad Y=X\cdot H,\quad y_{\text{fft}}=\mathrm{IDFT}_L(Y).
\]
Compute the linear convolution \(y_{\text{lin}}=x*h\) and keep the first \(N+M-1=2176\) samples. 
\textbf{Instruction:} Compare \(y_{\text{fft}}[0:2176]\) with \(y_{\text{lin}}[0:2176]\). They should match to within numerical precision. Do \emph{not} use \texttt{fftshift} for this check.
