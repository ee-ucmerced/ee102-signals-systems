\makeatletter
\def\input@path{{../styles/}{../../styles/}{../../../styles/}{../}{../../}{../../../}}
\makeatother

\documentclass{ee102_pset}
% macros.tex - Course meta information
\renewcommand{\course}{EE 102}
\renewcommand{\coursetitle}{Signal Processing and Linear Systems}
\renewcommand{\instructor}{Ayush Pandey}
\renewcommand{\semester}{Fall}
\renewcommand{\year}{2025}
\renewcommand{\shorttitle}{Week 1: Introduction to Signals}
% Use \renewcommand to avoid 'already defined' errors

% The following packages can be found on http:\\www.ctan.org
% \usepackage{graphics} % for pdf, bitmapped graphics files
%\usepackage{epsfig} % for postscript graphics files
%\usepackage{mathptmx} % assumes new font selection scheme installed
%\usepackage{times} % assumes new font selection scheme installed
\usepackage{amsmath} % assumes amsmath package installed
\usepackage{amssymb,mathtools}  % assumes amsmath package installed
\usepackage{xcolor}
\usepackage{pgfplots,subcaption}
\usepackage[hidelinks]{hyperref}
\usepackage{verbatim}
\usepackage{graphicx}
\usepackage{listings}

% -------- listings (Python) ----------
\lstdefinestyle{py}{
  language=Python,
  basicstyle=\ttfamily\small,
  keywordstyle=\color{blue!60!black}\bfseries,
  commentstyle=\color{green!40!black},
  stringstyle=\color{orange!60!black},
  showstringspaces=false,
  columns=fullflexible,
  frame=single,
  framerule=0.3pt,
  numbers=left,
  numberstyle=\tiny,
  xleftmargin=1em,
  tabsize=2,
  breaklines=true,
}
\usepackage[american]{circuitikz}
\usepackage{tikz}
\usepackage{caption}    
\usepackage{lscape}
\usepackage{soul}
\usepackage{tikz}
\usetikzlibrary{calc,angles,quotes,arrows.meta}

\usepackage{hyperref}
\hypersetup{
    colorlinks=true,
    linkcolor=blue,
    filecolor=magenta,      
    urlcolor=blue,
    pdftitle={week1_notes},
    pdfpagemode=FullScreen,
}
%\usepackage{float} 

%\usepackage[demo]{graphicx}
\pgfplotsset{compat=1.18}
% \usepgfplotslibrary{fillbetween}

\newsavebox{\measurebox}

\let\proof\relax\let\endproof\relax


\newcommand{\norm}[1]{\left\lVert#1\right\rVert}
\def\abs#1{\left\lvert#1\right\rvert}
\let\proof\relax
\let\endproof\relax
\usepackage{amsthm}
\usepackage{accents}
\usepackage{relsize}
\newcommand{\ubar}[1]{\underaccent{\bar}{#1}}
\newtheorem{theorem}{Theorem}
\newtheorem{corollary}{Corollary}[theorem]
\newtheorem{lemma}{Lemma}
\newtheorem{proposition}{Proposition}
\newtheorem{statement}{Statement}

\theoremstyle{definition}
\newtheorem{definition}{Definition}
 
\theoremstyle{remark}
\newtheorem*{remark}{Remark}
\theoremstyle{remark}
\newtheorem*{claim}{Claim}
\setlength{\parindent}{0cm}
\newenvironment{nalign}{
    \begin{equation}
    \begin{aligned}
}{
    \end{aligned}
    \end{equation}
    \ignorespacesafterend
}


% Assignment info
\author{\rule{3cm}{0.4pt}} % Name placeholder
\submitdate{\rule{3cm}{0.4pt}} % Submission date placeholder
\problemset{Homework \#11: Sampling}
% renew instructor to be empty
\renewcommand{\instructor}{}
\renewcommand{\duedate}{November 24, 2025}
\shorttitle{Homework \#11}
\begin{document}

\problem{1}
You know the following properties of a continuous-time signal \( x(t) \):
\begin{itemize}
    \item It is dominated by five main frequency components: 100 Hz, 300 Hz, 600 Hz, 900 Hz, and 1200 Hz, with other frequencies between 100 Hz and 1500 Hz having negligible amplitudes.
    \item The maximum amplitude of the signal in time-domain is 2.
    \item It is known that the signal can be uniquely reconstructed from its samples with a sampling frequency of 10,000 Hz.
\end{itemize}
\problempart [5 points]
Using the information provided, sketch the frequency domain representation \( X(\omega) \). You will not be able to draw an exact sketch. 

\problempart [10 points]
For what values of the sampling frequency \( f_s \) (in Hz) is $X(\omega)$ guaranteed to be zero? 

\problempart [5 points]
What is the Nyquist sampling rate for this signal? Justify your answer.

\problempart [5 points]
Sketch a block diagram that shows the process of sampling and perfect reconstruction of the signal \( x(t) \). Label all important components in your diagram and label the signal properties (e.g., sampling frequency, bandwidth, etc.) at each stage of the diagram.

\problempart [5 points]
By exploring the musical notes or human voice frequency ranges, suggest a real-world signal that could have similar frequency characteristics as the signal described above.

\problem{2} [Adapted from Problem 7.2 in Oppenheim \& Willsky]

A signal $x(t)$ with Fourier transform $X(j\omega)$ undergoes impulse-train sampling (like what we discussed during the lectures) to generate
\[
x_p(t) = \sum_{n=-\infty}^{\infty} x(nT)\,\delta(t-nT)
\]
where $T = 10^{-4}$. For each of the following sets of constraints on $x(t)$ and/or $X(j\omega)$, does the sampling theorem guarantee
that $x(t)$ can be recovered exactly from $x_p(t)$? Clearly justify your answers.

\textbf{[5 points each]}

\problempart $X(j\omega) = 0$ for $\lvert \omega \rvert > 5000\pi$
\problempart $X(j\omega) = 0$ for $\lvert \omega \rvert > 15000\pi$
\problempart $\Re\{X(j\omega)\} = 0$ for $\lvert \omega \rvert > 5000\pi$
\problempart $x(t)$ real and $X(j\omega) = 0$ for $\omega < -5000\pi$
\problempart $x(t)$ real and $X(j\omega) = 0$ for $\omega < -15000\pi$
\problempart $X(j\omega) * X(j\omega) = 0$ for $\lvert \omega \rvert > 15000\pi$
\problempart $\lvert X(j\omega) \rvert = 0$ for $\omega > 5000\pi$

\problem{3} [Adapted from Problem 7.22 in Oppenheim \& Willsky] 

The signal $y(t)$ is generated by convolving a band-limited signal
$x_1(t)$ with another band-limited signal $x_2(t)$, that is,
\[
y(t) = x_1(t) * x_2(t)
\]
where
\[
X_1(j\omega) = 0 \quad \text{for } \lvert \omega \rvert > 1000\pi,
\qquad
X_2(j\omega) = 0 \quad \text{for } \lvert \omega \rvert > 2000\pi.
\]

Impulse-train sampling is performed on $y(t)$ to obtain
\[
y_p(t) = \sum_{n=-\infty}^{\infty} y(nT)\,\delta(t-nT).
\]

\problempart \textbf{[15 points]} Specify the range of values for the sampling period $T$ which ensures that
$y(t)$ is recoverable from $y_p(t)$.
\problempart \textbf{[20 points]} Now, for a different pair of signals, $x_1(t) = \cos(2\pi t) + \cos(20 \pi t)$ and $x_2(t) = \text{sinc}(\pi t)$ (where $\text{sinc}(t) = \frac{\sin(t)}{t}$, with $\text{sinc}(0) = 1$), write a computer program (in Python or MATLAB) to verify the theoretically derived range of values for $T$ that ensure perfect reconstruction of $y(t)$ from $y_p(t)$. Then, visually show how the reconstruction of $y(t)$ from $y_p(t)$ fails when you have higher values of $T$ (i.e., lower sampling frequency) and succeeds when you have lower values of $T$ (i.e., higher sampling frequency). Include your code and plots in your submission.

\end{document}
