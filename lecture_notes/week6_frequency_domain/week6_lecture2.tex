\makeatletter
\def\input@path{{../styles/}{../../styles/}{../../../styles/}{../}{../../}{../../../}}
\makeatother
\documentclass{ee102_notes}
% macros.tex - Course meta information
\renewcommand{\course}{EE 102}
\renewcommand{\coursetitle}{Signal Processing and Linear Systems}
\renewcommand{\instructor}{Ayush Pandey}
\renewcommand{\semester}{Fall}
\renewcommand{\year}{2025}
\renewcommand{\shorttitle}{Week 1: Introduction to Signals}
% Use \renewcommand to avoid 'already defined' errors

% The following packages can be found on http:\\www.ctan.org
% \usepackage{graphics} % for pdf, bitmapped graphics files
%\usepackage{epsfig} % for postscript graphics files
%\usepackage{mathptmx} % assumes new font selection scheme installed
%\usepackage{times} % assumes new font selection scheme installed
\usepackage{amsmath} % assumes amsmath package installed
\usepackage{amssymb,mathtools}  % assumes amsmath package installed
\usepackage{xcolor}
\usepackage{pgfplots,subcaption}
\usepackage[hidelinks]{hyperref}
\usepackage{verbatim}
\usepackage{graphicx}
\usepackage{listings}

% -------- listings (Python) ----------
\lstdefinestyle{py}{
  language=Python,
  basicstyle=\ttfamily\small,
  keywordstyle=\color{blue!60!black}\bfseries,
  commentstyle=\color{green!40!black},
  stringstyle=\color{orange!60!black},
  showstringspaces=false,
  columns=fullflexible,
  frame=single,
  framerule=0.3pt,
  numbers=left,
  numberstyle=\tiny,
  xleftmargin=1em,
  tabsize=2,
  breaklines=true,
}
\usepackage[american]{circuitikz}
\usepackage{tikz}
\usepackage{caption}    
\usepackage{lscape}
\usepackage{soul}
\usepackage{tikz}
\usetikzlibrary{calc,angles,quotes,arrows.meta}

\usepackage{hyperref}
\hypersetup{
    colorlinks=true,
    linkcolor=blue,
    filecolor=magenta,      
    urlcolor=blue,
    pdftitle={week1_notes},
    pdfpagemode=FullScreen,
}
%\usepackage{float} 

%\usepackage[demo]{graphicx}
\pgfplotsset{compat=1.18}
% \usepgfplotslibrary{fillbetween}

\newsavebox{\measurebox}

\let\proof\relax\let\endproof\relax


\newcommand{\norm}[1]{\left\lVert#1\right\rVert}
\def\abs#1{\left\lvert#1\right\rvert}
\let\proof\relax
\let\endproof\relax
\usepackage{amsthm}
\usepackage{accents}
\usepackage{relsize}
\newcommand{\ubar}[1]{\underaccent{\bar}{#1}}
\newtheorem{theorem}{Theorem}
\newtheorem{corollary}{Corollary}[theorem]
\newtheorem{lemma}{Lemma}
\newtheorem{proposition}{Proposition}
\newtheorem{statement}{Statement}

\theoremstyle{definition}
\newtheorem{definition}{Definition}
 
\theoremstyle{remark}
\newtheorem*{remark}{Remark}
\theoremstyle{remark}
\newtheorem*{claim}{Claim}
\setlength{\parindent}{0cm}
\newenvironment{nalign}{
    \begin{equation}
    \begin{aligned}
}{
    \end{aligned}
    \end{equation}
    \ignorespacesafterend
}

\renewcommand{\releasedate}{October 8, 2025}

\newcommand{\Eblank}{\rule{3cm}{0.4pt}}
\newcommand{\Rankblank}{\rule{3cm}{0.4pt}}

\newcommand{\uof}[1]{u\!\left[#1\right]} % discrete-time unit step

\begin{document}

\section*{EE 102 Week 6, Lecture 2 (Fall 2025)}
\subsection*{Instructor: \instructor}
\subsection*{Date: \releasedate}

\section{Introduction and Review}
At the end of the previous lecture, we established that $e^{j\omega t}$ is an eigenfunction of LTI systems. That is, for an input $x(t)=e^{j\omega t}$, the output $y(t)$ is also a complex exponential at the same frequency $\omega$ but scaled by a complex number $H(j\omega)$:
\[
y(t)=H(j\omega)e^{j\omega t}.
\]
where 
\[
H(j\omega)=\int_{-\infty}^{\infty} h(\tau)\,e^{-j\omega \tau}\,d\tau.
\]
As a result of this \emph{very important} result, we proposed that if we are able to write any signal $x(t)$ as a linear combination of complex exponentials, then we can find the output $y(t)$ of an LTI system by simply scaling each complex exponential by $H(j\omega)$ and adding them up! 

The previous line is a one-line summary of Fourier analysis and synthesis --- something that we will be spending a lot of time on in the next few weeks.

We start this journey by positing the following problem:  Given a $T$-periodic signal $x(t)$, can we write it as a linear combination of complex exponentials? If so, how? More concretely, can we write
\[
x(t)=\sum_{k=-\infty}^{\infty} a_k\,e^{jk\omega_0 t}, 
\qquad \omega_0=\tfrac{2\pi}{T},
\]
the complex Fourier series coefficients are $\{a_k\}$. Note that $a_k \in \mathbb{C}$ are complex numbers, in general. The big question is; how do we find $a_k$?

\section{Goals}
Represent any periodic signal $x(t)$ as a linear combination of complex exponentials:
\begin{equation}
x(t)=\sum_{k=-\infty}^{\infty} a_k\,e^{jk\omega_0 t}.
\label{eq:fourier-series}
\end{equation}
Find the coefficients $a_k$.
\section{Introduction to Fourier Series}
Since the first part of the goal is ``any periodic signal $x(t)$'' and we are claiming that $x(t)$ can be written as a linear combination of complex exponentials. So, it must be true that (and we should make sure of it) that $e^{jk\omega_0 t}$ is periodic for every integer $k$.
\begin{popquiz}
Prove that (a) $e^{j\omega_0 t}$, (b) $e^{jk\omega_0 t}$ for $k \in \mathbb{Z}$, and (c) $\displaystyle\sum_{k=-\infty}^{\infty} e^{jk\omega_0 t}$ are all periodic and find their fundamental periods.
\popqsplit
We can just consider the general case: for $x(t) = e^{jk\omega_0 t}$ to be periodic with period $T$ we need
$x(t+T)=x(t)$, i.e.,
\[
e^{jk\omega_0 (t+T)}=e^{jk\omega_0 t}
\;\Longleftrightarrow\;
e^{jk\omega_0 T}=1 
\;\Longleftrightarrow\; k\omega_0 T=2\pi m,\; m\in\mathbb{Z}.
\]
Thus any $T=\dfrac{2\pi m}{k\omega_0}$ is a period and the smallest period (the fundamental period) is
\[
T_0=\frac{2\pi}{|k|\omega_0}.
\]
Since each harmonic $e^{jk\omega_0 t}$ has a period that is an integer divisor of $\dfrac{2\pi}{\omega_0}$, the sum of all harmonics is periodic with the common (fundamental) period
\[
T_0=\frac{2\pi}{\omega_0}.
\]
For $k=1$, we get the simpler result for $e^{j\omega_0 t}$.

Moreover, note that $e^{jk 2\pi}= \cos(2\pi k)+j\sin(2\pi k)=1$ for every integer $k$, confirming periodicity.
\end{popquiz}

Let's start to answer the second part of the goal: how do we find $a_k$? We will start with a simple example.
\subsection{Example: A mix-sinusoidal audio signal}
Consider the signal below that represents a combination of three sinusoids (added together). When you play a note of music at one specific frequency, you are playing one sinusoid. When you play a chord, you are playing multiple sinusoids at the same time by combining them together. So, in this example, we are representing an audio chord as a linear combination of complex exponentials to start our journey of representing \emph{any} periodic signal as a linear combination of complex exponentials.

Consider
\[
x(t)=\sin(6t)+\cos(2t)+\sin(12t),\qquad \omega_0=2.
\]
Write $x(t)$ as a linear combination of $e^{jk\omega_0 t}$:
\[
\sin(6t)=\frac{1}{2j}\!\left(e^{j6t}-e^{-j6t}\right)
= \frac{1}{2j}\!\left(e^{j(3)\omega_0 t}-e^{-j(3)\omega_0 t}\right),
\]
\[
\cos(2t)=\frac{1}{2}\!\left(e^{j2t}+e^{-j2t}\right)
=\frac{1}{2}\!\left(e^{j(1)\omega_0 t}+e^{-j(1)\omega_0 t}\right),
\]
\[
\sin(12t)=\frac{1}{2j}\!\left(e^{j12t}-e^{-j12t}\right)
= \frac{1}{2j}\!\left(e^{j(6)\omega_0 t}-e^{-j(6)\omega_0 t}\right).
\]
Hence
\[
x(t)=\sum_{k=-6}^{6} a_k\,e^{jk\omega_0 t},\qquad \omega_0=2,
\]
with the nonzero Fourier series coefficients as
\[
\begin{aligned}
a_{\pm 2}=\frac{1}{2},\qquad a_{\pm 3}=\pm \frac{1}{2j},a_{\pm 6}=\pm \frac{1}{2j},
\end{aligned}
\]
where the ``$\pm$'' pairs obey $a_{-k}=a_k^\ast$ for this real $x(t)$. 

Here, the $\cos$ term contributes the even coefficients $a_{\pm2}$; the $\sin$ terms contribute the odd, purely imaginary coefficients at $k=\pm3,\pm6$. All other $a_k$ are zero.

\section{The Trigonometric Form of Fourier Series}
If the linear combination form in equation~\eqref{eq:fourier-series} is confusing and the fact that ``we are representing everything as a linear combination of sinusoids'' is not obvious to you, you can see how we can rewrite the Fourier series synthesis equation~\eqref{eq:fourier-series} in a more familiar trigonometric form. Although you might not find the formulation below much useful, it will at least convince you that we are indeed representing everything as a linear combination of sinusoids.
\subsection{From exponentials to trigonometry}

For real $x(t)$, $x^\ast(t)=x(t)$, and
\[
x(t)=\sum_{k=-\infty}^{\infty} a_k e^{jk\omega_0 t}
=\sum_{k=1}^{\infty}\!\left( a_k e^{jk\omega_0 t}+a_{-k} e^{-jk\omega_0 t}\right)+a_0.
\]
Since $x(t)$ is real, we must have $a_{-k}=a_k^\ast$ (you can see that this is indeed the case in the example above with real signals). Therefore
\[
x(t)=a_0+\sum_{k=1}^{\infty}\!\left[a_k e^{jk\omega_0 t}+a_k^\ast e^{-jk\omega_0 t}\right]
= a_0+ 2\sum_{k=1}^{\infty} \Re\!\big\{a_k e^{jk\omega_0 t}\big\},
\]
where we used the fact that for any complex number $z$, $z+z^\ast=2\Re\{z\}$.

Writing $a_k=B_k+jC_k$ with $B_k,C_k\in\mathbb{R}$, we obtain the trigonometric form
\begin{equation}
\,x(t)=a_0+2\sum_{k=1}^{\infty}\!\big[\,B_k\cos(k\omega_0 t)-C_k\sin(k\omega_0 t)\,\big].\,
\label{eq:trig-form}
\end{equation}

Equation~\eqref{eq:trig-form} shows that $x(t)$ is a linear combination of sinusoids at frequencies $k\omega_0$, $k=1,2,\ldots$ with real coefficients. The constant term $a_0$ is the DC component (average value) of $x(t)$ (as we will see again in the next section). It's also finally an equation without any complex numbers or the imaginary term $j$ in it! So, it's hopefully more intuitive now. Let's continue towards our main goal --- finding $a_k$.

\subsection{The Fourier coefficients}
To find $a_k$ generally, let's start by multiplying both sides of equation~\eqref{eq:fourier-series} by $e^{-jn\omega_0 t}$ for some integer $n \in \mathbb{Z}$, we get
\[
x(t)\,e^{-jn\omega_0 t}
=\sum_{k=-\infty}^{\infty} a_k\,e^{jk\omega_0 t}\,e^{-jn\omega_0 t}
=\sum_{k=-\infty}^{\infty} a_k\,e^{j(k-n)\omega_0 t}.
\]
Next, let us integrate this equation over $[0,T)$,
\[
\int_{0}^{T} x(t)\,e^{-jn\omega_0 t}\,dt
=\sum_{k=-\infty}^{\infty} a_k \int_{0}^{T} e^{j(k-n)\omega_0 t}\,dt
\]
Now, we need to evaluate the integral on the right-hand side. We have two cases:
\begin{itemize}
\item If $k=n$, then 
\[ \int_{0}^{T} e^{j(k-n)\omega_0 t}\,dt=\int_{0}^{T} 1\,dt=T\] since $e^{j0}=1$.

\item If $k\neq n$, then we have 
\[
\int_{0}^{T} e^{j(k-n)\omega_0 t}\,dt
\]
Using orthogonality over one period $T$, this integral is 0 (since integrating a sinusoid over one period will lead to the positive and negative areas canceling out). You can verify this by direct integration too:
\[
\int_{0}^{T} e^{j(k-n)\omega_0 t}\,dt
=\left[\frac{e^{j(k-n)\omega_0 t}}{j(k-n)\omega_0}\right]_{0}^{T}
=\frac{e^{j(k-n)\omega_0 T}-1}{j(k-n)\omega_0}=0
\]
since $e^{j(k-n)\omega_0 T}=e^{j(k-n)2\pi}=1$ for every integer $k-n$ (from the pop-quiz above).
\end{itemize}

Hence, we have the Fourier series analysis equation (the equation that gives us values of $a_k$):
\[
\,a_n=\frac{1}{T}\int_{0}^{T} x(t)\,e^{-jn\omega_0 t}\,dt\,
\qquad n\in\mathbb{Z},
\]
if you replace $n$ by $k$, you get the more familiar form:
\[
a_k=\frac{1}{T}\int_{0}^{T} x(t)\,e^{-jk\omega_0 t}\,dt,
\qquad k\in\mathbb{Z},
\]
which is the formula for the Fourier series coefficients.
Note that for $k=0$, we have
\[
a_0=\frac{1}{T}\int_{0}^{T} x(t)\,dt,
\]
which is the average value (or DC component) of $x(t)$ over one period.

\subsection{Properties of Fourier Series coefficients}
There are many helpful properties that you should know about Fourier series coefficients. Here are a few of them (you can find the full list in the textbook):

\paragraph{Linearity}

For two periodic signals $x(t)$ and $y(t)$ with Fourier Series coefficients $a_k$ and $b_k$, if we construct another signal by linear superposition, $z(t)=A x(t)+B y(t)$, then the Fourier Series coefficients for $z(t)$ satisfy
\[
z(t)\;\Longleftrightarrow\; \{\,A a_k + B b_k\,\}_{k\in\mathbb{Z}}.
\]

\paragraph{Periodic convolution}
The Fourier series coefficients for the output $y(t)$ of a system with impulse response $h(t)$ to a periodic input $x(t)$ can be computed using periodic convolution.

Let $y(t)=(x*h)(t)$ denote periodic convolution with period $T$.
We write the Fourier series expansion of $x(t)$ and $h(t)$ as
\[
x(t)=\sum_{\ell=-\infty}^{\infty} a_\ell e^{j\ell\omega_0 t},
\qquad
h(t)=\sum_{m=-\infty}^{\infty} b_m e^{jm\omega_0 t},
\qquad \omega_0=\tfrac{2\pi}{T}.
\]
Let $y(t)=(x*h)(t)$ denote the periodic convolution (integrate only for one period):,
\[
y(t)=\int_{0}^{T} x(\tau)\,h(t-\tau)\,d\tau .
\]
Let $\{c_k\}$ be the Fourier Series coefficients of $y(t)$, defined as
\[
c_k=\frac1T \int_{0}^{T} y(t)\,e^{-jk\omega_0 t}\,dt .
\]

We can find $c_k$ in terms of $a_k$ and $b_k$ using convolution as follows. Start by substituting the expression for $y(t)$ into the definition of $c_k$:
\[
c_k=\frac1T\int_{0}^{T} \left[\int_{0}^{T} x(\tau)\,h(t-\tau)\,d\tau\right]
e^{-jk\omega_0 t}\,dt .
\]

Now, using the following equations for the Fourier Series expansions of $x(\tau)$ and $h(t-\tau)$ (this is also an in-place proof for the time-shift property of Fourier Series!):

\[
x(\tau)=\sum_{\ell} a_\ell e^{j\ell\omega_0 \tau}
\]
and
\[
h(t-\tau)=\sum_{m} b_m e^{jm\omega_0 (t-\tau)}
= \sum_{m} b_m e^{jm\omega_0 t}\,e^{-jm\omega_0 \tau},
\]
we get
\[
\begin{aligned}
c_k
&=\frac1T\int_{0}^{T}\!\!\int_{0}^{T}
\left(\sum_{\ell} a_\ell e^{j\ell\omega_0 \tau}\right)
\left(\sum_{m} b_m e^{jm\omega_0 t}e^{-jm\omega_0 \tau}\right)
e^{-jk\omega_0 t}\,d\tau\,dt .
\end{aligned}
\]

Interchange sums and integrals and collect factors together to write,
\[
\begin{aligned}
c_k
&=\frac1T\sum_{\ell}\sum_{m} a_\ell b_m
\int_{0}^{T}\!\!\int_{0}^{T}
e^{j\ell\omega_0 \tau} \, e^{-jm\omega_0 \tau}\,
e^{jm\omega_0 t}\,e^{-jk\omega_0 t}\, d\tau\,dt \\
&=\frac1T\sum_{\ell}\sum_{m} a_\ell b_m
\left[\int_{0}^{T} e^{j(m-k)\omega_0 t}\,dt\right]
\left[\int_{0}^{T} e^{j(\ell-m)\omega_0 \tau}\,d\tau\right].
\end{aligned}
\]

Finally, note that periodic integral over one period is 0 unless the integrand is constant. So, for any integers $p$,
\(
\int_{0}^{T} e^{jp\omega_0 t}\,dt
=\begin{cases}
T, & p=0,\\
0, & p\neq 0.
\end{cases}
\)
Hence the $t$-integral is zero unless $m=k$, and the $\tau$-integral is zero
unless $\ell=m$:
\[
\int_{0}^{T} e^{j(m-k)\omega_0 t}\,dt = T\,\delta_{m,k},
\qquad
\int_{0}^{T} e^{j(\ell-m)\omega_0 \tau}\,d\tau = T\,\delta_{\ell,m}.
\]

So, only the terms with $\ell=m=k$ survive!
\[
c_k=\frac1T \sum_{\ell}\sum_{m} a_\ell b_m \,(T\delta_{m,k})\,(T\delta_{\ell,m})
=\frac1T \,(T)(T)\, a_k b_k
= T\,a_k b_k .
\]

\[
c_k = T\,a_k\,b_k\,,\qquad k\in\mathbb{Z}.
\]

Thus, periodic convolution corresponds to a \emph{line-by-line} product of Fourier Series
coefficients: each harmonic $k$ of the output equals $T$ times the product of
the input and impulse response harmonics at the same $k$.



\paragraph{Filtering of output using aperiodic impulse response}

If $h(t)$ is aperiodic (but LTI) and $x(t)$ is $T$-periodic with Fourier Series coefficients $\{a_k\}$, then
\[
y(t)=(x*h)(t)=\int_{-\infty}^{\infty} h(\tau)\,x(t-\tau)\,d\tau
= \sum_{k} a_k e^{jk\omega_0 t}\,\underbrace{\int_{-\infty}^{\infty} h(\tau)\,e^{-jk\omega_0 \tau}\,d\tau}_{\displaystyle H(jk\omega_0)}.
\]
Hence
\[
\,y(t)=\sum_{k=-\infty}^{\infty} a_k\,H(jk\omega_0)\,e^{jk\omega_0 t}\,
\]
i.e., each harmonic is scaled by the continuous-time frequency response $H(j\omega)$ evaluated at $\omega= k\omega_0$. So, the Fourier Series coefficients of $y(t)$ are
\[
c_k=a_k\,H(jk\omega_0).
\]
This will be very useful for your homework problems!

\section{Practice Problems}
\begin{enumerate}
    \item Solved Example 3.6 in Oppenheim and Willsky (2nd Edition) --- the square wave
    \item Solved Example 3.7 in Oppenheim and Willsky (2nd Edition) --- the ramp function
    \item Work through the properties in Table 3.1 in Oppenheim and Willsky (2nd Edition)
    \item Solved Example 3.5 in Oppenheim and Willsky (2nd Edition) --- the square wave \textbf{(this is similar to HW 6 problem 1 and 2!)}
\end{enumerate}
\end{document}
