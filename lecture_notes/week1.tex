\documentclass{ee102_notes}
% macros.tex - Course meta information
\renewcommand{\course}{EE 102}
\renewcommand{\coursetitle}{Signal Processing and Linear Systems}
\renewcommand{\instructor}{Ayush Pandey}
\renewcommand{\semester}{Fall}
\renewcommand{\year}{2025}
\renewcommand{\shorttitle}{Week 1: Introduction to Signals}
% Use \renewcommand to avoid 'already defined' errors

\usepackage{amsmath}
\usepackage{amssymb}
\usepackage{graphicx}

\begin{document}

\section*{EE102 Week 1, Lecture 1 (Fall 2025)}
\subsection*{Instructor: Ayush Pandey}
\subsection*{Date: Aug 27, 2025}
\section{Goals}
\begin{itemize}
  \item Logistics, grading, extensions, expectations
  \item Motivation to study signal processing
\end{itemize}

\section{Pre-requisite \#1: Vectors}
When studying problems with many entities/observations, we structure our variables into vectors.

An $n$-dimensional vector $\mathbf{x}$ can be written as
\[
\mathbf{x}=[\,x_1,\;x_2,\;\ldots,\;x_n\,], \qquad \mathbf{x}\in\mathbb{R}^n .
\]

\subsection*{Matrices are transformations}
If you transform a vector $\mathbf{x}$ to a new vector $\mathbf{y}$ such that all elements in $\mathbf{y}$ are linear combinations of elements in $\mathbf{x}$, then the transformation is called a matrix.

\[
\mathbf{x}=
\begin{bmatrix}
x_1\\ x_2\\ \vdots\\ x_n
\end{bmatrix}
\;\longmapsto\;
\mathbf{y}=
\begin{bmatrix}
y_1\\ y_2\\ \vdots\\ y_m
\end{bmatrix}
\]
such that
\[
y_1=\sum_{i=1}^{n}\alpha_{1i}x_i,\qquad
y_2=\sum_{i=1}^{n}\alpha_{2i}x_i,\quad \ldots,\quad
y_m=\sum_{i=1}^{n}\alpha_{mi}x_i .
\]
Then $A\mathbf{x}=\mathbf{y}$, where
\[
A\in\mathbb{R}^{m\times n}, \qquad
A=
\begin{bmatrix}
\alpha_{11} & \alpha_{12} & \cdots & \alpha_{1n}\\
\alpha_{21} & \alpha_{22} & \cdots & \alpha_{2n}\\
\vdots      & \vdots      & \ddots & \vdots\\
\alpha_{m1} & \alpha_{m2} & \cdots & \alpha_{mn}
\end{bmatrix}.
\]

We write
\[
A:\; X \to Y,
\]
where $X$ is the vector space in $\mathbb{R}^n$ where $\mathbf{x}$ lies and $Y$ is the vector space in $\mathbb{R}^m$ where $\mathbf{y}$ lies.

\paragraph{Recall}
\begin{itemize}
  \item Diagonal matrix
  \item Identity matrix
  \item Symmetric matrix
  \item Zero matrix
  \item Matrix transpose
  \item Matrix algebra ($+$, $-$, $\times$, inverse)
\end{itemize}
\section{Pre-requisite \#2: Complex numbers}

Although the usual way we learn about the complex unit ``$\mathrm{j}$'' is as
a convenient notation for a solution of
\[
x^{2}+1=0 \;\;\Rightarrow\;\; x=\sqrt{-1}:=\mathrm{j},
\]
it is useful to recognize other places where this convenience is beneficial.
In signal processing we are often looking for easy ways to analyze physical
signals, not only to solve algebraic equations.

\paragraph{From vectors to a complex scalar.}
Given a vector
\[
\mathbf{x}=\begin{bmatrix}x_1\\ x_2\end{bmatrix},
\]
define the (complex) scalar
\[
z_{\mathbf{x}}=x_1+\mathrm{j}x_2.
\]
The entries $x_1$ and $x_2$ are not ``added'' in $\mathbb{R}$; they are bound
only because they are components of the same vector. Writing $z_{\mathbf{x}}$
lets us treat the vector like a single scalar living in $\mathbb{C}$.

\paragraph{Inner product via complex numbers.}
For $\mathbf{x}=[x_1,x_2]^{\top}$ and $\mathbf{y}=[y_1,y_2]^{\top}$,
\begin{align*}
\langle \mathbf{x},\mathbf{y}\rangle
&= \mathbf{x}^{\top}\mathbf{y}
= x_1y_1+x_2y_2 .
\end{align*}
With the complex representations
\[
z_{\mathbf{x}}=x_1+\mathrm{j}x_2,\qquad
z_{\mathbf{y}}=y_1+\mathrm{j}y_2,
\]
their product with conjugation is
\begin{align*}
z_{\mathbf{x}}\;\overline{z_{\mathbf{y}}}
&=(x_1+\mathrm{j}x_2)(y_1-\mathrm{j}y_2) \\
&=(x_1y_1+x_2y_2)\;+\;\mathrm{j}\,(-x_1y_2+x_2y_1).
\end{align*}
Taking the real part gives the vector inner product:
\[
\Re\!\left(z_{\mathbf{x}}\;\overline{z_{\mathbf{y}}}\right)=x_1y_1+x_2y_2
=\langle \mathbf{x},\mathbf{y}\rangle .
\]

\paragraph{Polar form viewpoint.}
Write $\mathbf{x}$ in polar coordinates with $r_x=\|\mathbf{x}\|$ and angle
$\theta_x$:
\[
x_1=r_x\cos\theta_x,\qquad x_2=r_x\sin\theta_x,
\]
so
\[
z_{\mathbf{x}}=r_x(\cos\theta_x+\mathrm{j}\sin\theta_x)=r_x e^{\mathrm{j}\theta_x}.
\]
Similarly $z_{\mathbf{y}}=r_y e^{\mathrm{j}\theta_y}$. Then
\begin{align*}
z_{\mathbf{x}}\;\overline{z_{\mathbf{y}}}
&= r_x e^{\mathrm{j}\theta_x}\; r_y e^{-\mathrm{j}\theta_y}
= r_x r_y e^{\mathrm{j}(\theta_x-\theta_y)} \\
&= r_x r_y\Big[\cos(\theta_x-\theta_y)
  + \mathrm{j}\sin(\theta_x-\theta_y)\Big],
\end{align*}
hence
\[
\Re\!\left(z_{\mathbf{x}}\;\overline{z_{\mathbf{y}}}\right)
= r_x r_y \cos(\theta_x-\theta_y)
= \langle \mathbf{x},\mathbf{y}\rangle .
\]

You can make intuitive sense of the  inner product in polar form by understanding that the angle between the two vectors is $\theta_x - \theta_y$ (plot the two vectors in a 2D plane). Therefore, the scalar projection (see notes below) of $\textbf{x}$ onto $\textbf{y}$ is $r_x \cos(\theta_x - \theta_y)$, which when multiplied by the absolute value of $y$ gives the inner product (that is, $r_xr_y\cos(\theta_x - \theta_y)$ in complex polar form). 
\paragraph{Scalar projection.}
Let $\hat{\mathbf{y}}=\mathbf{y}/\|\mathbf{y}\|$ and let $\phi$ be the angle
between $\mathbf{x}$ and $\mathbf{y}$. Then
\[
\text{proj}_{\mathbf{y}}(\mathbf{x})=(\mathbf{x}\cdot\hat{\mathbf{y}})\,\hat{\mathbf{y}},
\qquad
\|\text{proj}_{\mathbf{y}}(\mathbf{x})\|
=\mathbf{x}\cdot\hat{\mathbf{y}}=\|\mathbf{x}\|\cos\phi,
\]
and
\[
\mathbf{x}\cdot\mathbf{y}
=\mathbf{x}\cdot\big(\|\mathbf{y}\|\hat{\mathbf{y}}\big)
=\|\mathbf{y}\|\;(\mathbf{x}\cdot\hat{\mathbf{y}}).
\]

% \section{}
\end{document}
